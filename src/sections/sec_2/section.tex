\section{Системное проектирование}
\label{sec:sysproj}

Установка требований к функционалу разрабатываемой в рамках
курсового проекта программе помогает провести разделение всего алгоритма
работы приложения на функциональные блоки. Функциональные блоки -- это
блоки программного компонента, которые ответственны за определенную
задачу, а совокупность функциональных блоков позволяет реализовать
полноценную работу программы. Наличие функциональных блоков упрощает 
архитектуру приложения и уменьшает количество времени на понимание 
внутреннего устройства программы, обеспечит гибкость и масштабируемость 
приложения с целью последующей возможной доработки путем добавления 
дополнительных программных блоков.


Программа анализатора сетевого трафика может быть разбита на модули,
каждый из которых отвечает за определенную функцию. В рамках данного
курсового проекта программа разделена на три основных части: модуль
расшифровки, модуль анализа пакетов, модуль сохранения и визуализации
данных. Каждый модуль, входящий в состав программы, играет важную роль в
работе анализатора сетевого трафика и выполняет определенную функцию,
влияющую на точность, эффективность и удобство использования
программы в целом. Все части программы взаимодействуют между собой, чтобы
улучшить работу сниффера и удовлетворить все запросы пользователя.
Графическая структура модулей программы представлена в приложении \hyperref[sec:appendix:scheme_struct]{А}.



\subsection{Модуль захвата, идентификации и подсчета пакетов}

Этот программный модуль нужен для выполнения захвата пакетов и 
идентификации их типов. Также модуль реализует подсчет общего 
количества анализируемых пакетов, что позволяет увидеть промежуточные данные.


В рамках модуля программа захватывает пакет при помощи сокета,
определяет типа пакета в зависимости от протокола, которому принадлежит
пакет. Определив тип блока данных, в модуле увеличивается счетчик 
соответствующего типа пакета. Затем полученные данные из других частей 
приложения выводятся в консоль.


\subsection{Модуль расшифровки заголовков пакетов}

Программный модуль, реализующий расшифровку IP Header и
расшифровку Ethernet Header для последующего парсинга пакетов и анализа
полученной информации.


В рамках модуля с помощью сокета открывается доступ к данным,
хранящимся внутри вышеуказанных структур, что, в свою очередь,
предоставляет возможность получения данных, необходимых для дальнейшей
работы программы.


В результате в модуле расшифровки можно выделить два подмодуля:
\begin{itemize}
    \item модуль расшифровки \texttt{IP Header};
    \item модуль расшифровки \texttt{Ethernet Header}.
\end{itemize}


\subsection{Модуль анализа протоколов}

Программный модуль, анализирующий все данные протокола пакета,
поступающие из трафика, и отбирающий только необходимые для работы
программы параметры, используя определенные критерии. Данные критерии
варьируются от пакета к пакету в зависимости от протокола.


Выделенный модуль выполняет декодирование сетевых протоколов,
извлечение данных из заголовков пакетов и анализ содержимого пакетов для
мониторинга трафика, выявления проблем в сети или аномалий в поведении
приложений.


Каждый протокол имеет свои поля данных, поэтому сниффер считывает
их в соответствии с заданными критериями, установленными непосредственно
программой.


В результате в модуле анализа протоколов можно выделить три
подмодуля:
\begin{itemize}
    \item модуль анализа TCP протокола;
    \item модуль анализа UDP протокола;
    \item модуль анализа ICMP протокола;
\end{itemize}


\subsection{Модуль сохранения и визуализации данных}

Программный модуль, позволяющий отображать полученную
информацию в удобном для пользователя формате, например, в виде графиков,
таблиц и диаграмм. В рамках данного курсового проекта программа
автоматически сохраняет информацию в текстовый файл в виде лога 
(журнала событий).


Таким образом, информация предоставляется в удобном пользователю
виде. Он может отследить все пакеты, которые были проанализированы
программы, изучить их заголовки и другие параметры обнаруженных
протоколов.
