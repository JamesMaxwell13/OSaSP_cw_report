\section{Результаты работы}
\label{sec:result}

\subsection{Тестирование}

Запуск программы производится через терминал из директории, в которой
находится исполняемый файл. Так как программе нужен доступ к raw сокетам, 
важно помнить, что само приложение необходимо запускать с правами пользователя 
root\cite{linux} или используя системную утлиту sudo:
\begin{verbatim}
    sudo ./main
\end{verbatim}


Сразу после запуска программа запускает процесс захвата и анализа
пакетов. Об этом сигнализирует строка консоли. В процессе выполнения в
консоль выводится промежуточное количество захваченных пакетов:
\begin{verbatim}
    Starting...
    TCP: 434 UDP: 77 ICMP: 0 Others: 71 Total: 582
\end{verbatim}


О завершении программы сигнализирует соответствующая строка
консоли, которая выводится сразу после вывода конечного количества
захваченных пакетов:
\begin{verbatim}
    Starting...
    TCP: 582 UDP: 101 ICMP: 0 Others: 76 Total: 759
    Finished
\end{verbatim}


Результаты работы программы сохраняются в текстовый файл, который
создается автоматически. Данные в файле отформатированы так, чтобы
пользователю было удобно их проанализировать. Каждый пакет имеет
заголовок, в котором указывается протокол. Далее размещаются IP и Ethernet
заголовки с описанными внутренними параметрами, после чего расписаны
параметры непосредственно протокола. В конце приводится блок данных,
которые находились в заголовках пакета в момент его захвата. Фрагмент файла
с результатами выглядит следующим образом:
\footnotesize
\begin{verbatim}
            **********************TCP Packet*************************

        Ethernet Header
        |-Destination Address : CC-03-FA-62-83-7D 
        |-Source Address : EC-2E-98-54-E2-75 
        |-Protocol : 8 

        IP Header
        |-IP Version : 4
        |-IP Header Length : 5 DWORDS or 20 Bytes
        |-Type Of Service : 0
        |-IP Total Length : 205 Bytes (Size of Packet)
        |-Identification : 7213
        |-TTL : 64
        |-Protocol : 6
        |-Checksum : 8311
        |-Source IP : 192.168.0.27
        |-Destination IP : 149.154.167.41

        TCP Header
        |-Source Port : 42178
        |-Destination Port : 443
        |-Sequence Number : 3223488899
        |-Acknowledge Number : 2709835259
        |-Header Length : 8 DWORDS or 32 BYTES
        |-Urgent Flag : 0
        |-Acknowledgement Flag : 1
        |-Push Flag : 1
        |-Reset Flag : 0
        |-Synchronise Flag : 0
        |-Finish Flag : 0
        |-Window : 4680
        |-Checksum : 14390
        |-Urgent Pointer : 0

        DATA Dump 
        IP Header
        CC 03 FA 62 83 7D EC 2E 98 54 E2 75 08 00 45 00 	...b.}...T.u..E.
        00 CD 1C 2D                                     	...-
        TCP Header
        40 00 40 06 20 77 C0 A8 00 1B 95 9A A7 29 A4 C2 	@.@. w.......)..
        01 BB C0 22 89 83 A1 84 CD FB 80 18 12 48 38 36 	..."......Ђ..H86
        Data Payload
        93 0D 55 41 36 8E 67 F8 F4 DD 7E 12 00 35 37 C0 	..UA6.g...~..57.
        D4 2F 16 8A 99 AC 8B 75 72 BD 55 06 06 A7 25 2D 	./.....ur.U...%-
        9F AC 0E 6A 3E EB 34 62 F4 2C 5A B3 D4 E4 61 81 	...j>.4b.,Z...a.
        B8 ED 19 3E F6 93 AD 8E 97 E3 6C 0C 5E C1 F0 84 	...>......l.^...
        C7 3F 1E 3C BB C2 6B B5 E9 97 E9 20 72 68 C8 2F 	.?.<..k.... rh./
        9B 70 CD DF 37 54 F6 71 F0 97 1E AB 93 D2 07 52 	.p..7T.q.......R
        00 4A 9F 93 5D 1A A4 C0 FD 30 EC 2A 21 AB 6B 3F 	.J..]....0.*!.k?
        C7 46 63 AB DB 23 12 0E 27 19 C6 DF 6F BC 62 3C 	.Fc..#..'...o.b<
        84 79 2F D6 8F A8 24 B2 9A 5C 9F 72 56 C4 23 96 	.y/...$..\.rV.#.
        E2 C8 54 74 E2 DE 8F 90 17                      	..Tt.....        
\end{verbatim}
\normalsize


В работе программы предусмотрены возможные исключения. Если при 
создании файла, в который будут заноситься данные, происходит ошибка, в 
консоль будет выведено соответствующее сообщение\cite{linux_prog}, и программа будет 
экстренно завершена:
\begin{verbatim}
    Starting…
    Unable to create log.txt file
\end{verbatim}


В случае неудачной попытки открыть сокет в консоль будет выведено
соответствующее сообщение об ошибке\cite{linux_prog}, и программа будет экстренно
завершена:
\begin{verbatim}
    Starting…
    Socket Error: Bad file descriptor
\end{verbatim}


При неудочной попытке чтения пакета в консоль будет выведено
соответствующее сообщение об ошибке\cite{linux_prog}, и программа будет экстренно
завершена:
\begin{verbatim}
    Starting…
    Recvfrom error , failed to get packets
\end{verbatim}


Если программа была запущена без прав привилегированного пользователя,
будет сгенерировано сообщение об ошибке доступа к сокету:
\begin{verbatim}
    Starting…
    Socket Error: Operation not permited     
\end{verbatim}


\subsection{Код программмы}
Код программы представлен в приложении \hyperref[sec:appendix:code]{Г}.
