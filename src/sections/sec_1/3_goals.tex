\subsection{Постановка задачи}

Найдя и исследовав множество аналогов программных инструментов,
которые предоставляют пользователю возможность мониторить и
анализировать сетевой трафик, можно сделать вывод, что сниффер требует
достаточно много ресурсов для того, чтобы предоставлять такое количество
информации. Также необходимо учитывать тот факт, что полученные данные 
необходимо сохранять, сортировать и предоставлять пользователю
в удобной для ознакомления и обработки форме.

Учитывая все вышеприведенные условия и особенности, в рамках данного
курсового проекта будет реализован анализатора сетевого трафика, или
сниффер, способный мониторить и оценивать сетевой трафик. Анализатор
должен обладать функциями выявления особенностей и сохранения трафика, 
а также генерации отчетов на основе полученных данных. 
Будет рассмотрена возможность визуализации данных в удобной для 
пользователя форме. Важными аспектами для разработки также являются 
эффективность работы и поддержка различных протоколов, таких как TCP/IP, UDP, ICMP и другие.

В результате работы должен быть создан инструмент, который позволит
эффективно мониторить и анализировать сетевой трафик в режиме реального
времени. Основная функциональность программы разработана с
использованием языка высокого уровня Си. В качестве операционной системы
была выбрана Ubuntu 22.04 LTS, которая базируется на ядре Linux версии 6.5.
Сетевое взаимодействие основано на сокетах, являющихся разновидностью
сокетов Беркли, в частности raw сокеты (от англ. raw -- сырой), с помощью
которых можно получить доступ к протоколам сетевого уровня. В рамках
данного курсового проекта сокеты работают с протоколами транспортного
уровня TCP/UDP, а с помощью возможностей raw сокета так же анализируют
протоколы сетевого уровня IP/ICMP.
