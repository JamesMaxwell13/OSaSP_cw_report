\sectionCenteredToc{Введение}
\label{sec:intro}

В терминологии системных администраторов и специалистов по
информационной безопасности часто встречается понятие -- «анализаторы
трафика». Под анализатором трафика понимается устройство или программа,
которая перехватывает трафик и затем его анализирует. Такие продукты полезны
в тех случаях, когда нужно извлечь из потока данных какие-либо сведения
(например, пароли) или провести диагностику сети. Современные коммерческие
анализаторы выпускаются как в виде программных решений, так и в виде
аппаратных устройств и служат для комплексного анализа производительности
крупных информационных сетей, а также пользовательских приложений.


Анализаторы сетевого трафика -- это новая категория продуктов
безопасности, которые используют сетевые коммуникации в качестве основного
источника данных для обнаружения и расследования угроз безопасности,
аномального или вредоносного поведения в сети. Анализаторы, или так
называемые снифферы (от англ. to sniff -- нюхать), передают всю
необходимую информацию о событиях, происходящих внутри сети, в центры
мониторинга и реагирования.


Есть несколько веских причин для мониторинга общего трафика в сети.
Информация, полученная с помощью инструментов мониторинга сетевого
трафика, позволяет администратору решать следующие задачи:
\begin{itemize}
    \item находить проблемы в работе сети (задержки в передаче информации) и 
    быстро их устранять;
    \item обнаруживать шпионские программы, взломы и другую постороннюю активность 
    (несанкционированный доступ злоумышленников);
    \item анализировать работоспособность пользовательских приложений;
    \item собирать статистику;
    \item определять ведомственную пропускную способность.
\end{itemize}

Сниффер может анализировать только то, что проходит через его сетевую
карту. Внутри одного сегмента сети Ethernet все пакеты рассылаются всем
машинам, из-за этого возможно перехватывать чужую информацию. Между
сегментами информация передаётся через коммутаторы. Использование
коммутаторов и их грамотная конфигурация уже является защитой от
прослушивания. Коммутация пакетов -- форма передачи, при которой данные,
разбитые на отдельные пакеты, могут пересылаться из исходного пункта в пункт
назначения разными маршрутами. Так что если кто-то в другом сегменте
посылает внутри него какие-либо пакеты, то в ваш сегмент коммутатор эти
данные не отправит.