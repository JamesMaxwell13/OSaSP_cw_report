\subsection{Описание основных структур данных программы}

Разработанная программа включает в себя следующие структуры данных:
\begin{enumerate}
    \item[1] {{\texttt{Struct sockaddr}} -- системная структура данных, описывающая сокет\cite{linux_prog}.
    Содержит следующие поля:
        \begin{itemize}
            \item \texttt{u{\textunderscore}short sa{\textunderscore}family}\cite{unix_net} -- код семейства адресов, указывающий на тип
            адреса в зависимости от протокола;
            \item \texttt{char sa{\textunderscore}data} -- массив, содержащий остальную часть адреса,
            необходимую для использования в конкретных сетевых протоколах.
        \end{itemize}}

    \item[2] {{\texttt{Struct sockaddr{\textunderscore}in}} -- системная структура данных, описывающая сокет
    в домене AF{\textunderscore}INET\cite{opennet}. Содержит следующие поля:
        \begin{itemize}
            \item \texttt{short sin{\textunderscore}family}\cite{unix_net} -- семейство адресов (по умолчанию
            устанавливается на значение AF{\textunderscore}INET);
            \item \texttt{u{\textunderscore}short sin{\textunderscore}port}\cite{linux_prog} -- номер порта, на который должен быть
            отправлен пакет;
            \item \texttt{struct in{\textunderscore}addr, sin{\textunderscore}addr}\cite{opennet} -- структуры, содержащие IP адрес, по
            которому должен быть отправлен пакет;
            \item \texttt{char sin{\textunderscore}zero} -- массив, используемый для дополнения структуры
            до размера структуры sockaddr.
        \end{itemize}}

    \item[3] {{\texttt{Struct iphdr}} -- системная структура данных, хранящая параметры IP
    заголовка. Содержит следующие поля:
        \begin{itemize}
            \item \texttt{unsigned int ihl} -- длина IP заголовка в 32-битных словах;
            \item \texttt{unsigned int version} -- версия протокола IP;
            \item \texttt{u{\textunderscore}int8{\textunderscore}t tos} -- тип сервиса (указания приоритета трафика);
            \item \texttt{u{\textunderscore}int16{\textunderscore}t tot{\textunderscore}len} -- общая длина IP пакета в байтах, включая
            заголовок и данные;
            \item \texttt{u{\textunderscore}int16{\textunderscore}t id} -- идентификатор IP пакета;
            \item \texttt{u{\textunderscore}int16{\textunderscore}t frag{\textunderscore}off} -- смещение фрагмента IP пакета в байтах от
            начала пакета;
            \item \texttt{u{\textunderscore}int8{\textunderscore}t ttl} -- время жизни (количество маршрутизаторов,
            которые пакет может проходить до его отбрасывания);
            \item \texttt{u{\textunderscore}int8{\textunderscore}t protocol} -- номер протокола верхнего уровня, который
            передает данные в IP пакете;
            \item \texttt{u{\textunderscore}int16{\textunderscore}t check} -- контрольная сумма заголовка IP пакета
            (проверка целостности пакета);
            \item \texttt{u{\textunderscore}int32{\textunderscore}t saddr} -- IP адрес отправителя пакета;
            \item \texttt{u{\textunderscore}int32{\textunderscore}t daddr} -- IP адрес получателя пакета.
        \end{itemize}}

    \item[4] {{\texttt{Struct ethhdr}} -- системная структура данных, хранящая параметры
    Ethernet заголовка. Содержит следующие поля:
        \begin{itemize}
            \item \texttt{unsigned char h{\textunderscore}dest[ETH{\textunderscore}ALEN]} -- массив байт длиной
            ETH{\textunderscore}ALEN\cite{unix_prof}, содержащий MAC-адрес получателя Ethernet фрейма;
            \item \texttt{unsigned char h{\textunderscore}source[ETH{\textunderscore}ALEN]} -- массив байт длиной
            ETH{\textunderscore}ALEN, содержащий MAC-адрес источника Ethernet фрейма;
            \item \texttt{be16 h{\textunderscore}proto} -- двухбайтовое значение, определяющее тип
            протокола, который используется внутри данных пакета.
        \end{itemize}}

    \item[5] {{\texttt{Struct tcphdr}} -- системная структура данных, хранящая параметры TCP
    протокола. Содержит следующие поля:
        \begin{itemize}
            \item \texttt{u{\textunderscore}int16{\textunderscore}t source} -- порт отправителя;
            \item \texttt{u{\textunderscore}int16{\textunderscore}t dest} -- порт получателя;
            \item \texttt{u{\textunderscore}int32{\textunderscore}t seq} -- номер последовательности для передачи данных;
            \item \texttt{u{\textunderscore}int32{\textunderscore}t ack{\textunderscore}seq} -- подтверждение получения
            последовательности;
            \item \texttt{u{\textunderscore}int16{\textunderscore}t doff} -- размер заголовка TCP протокола в 32-битных
            словах\cite{unix_prof};
            \item \texttt{u{\textunderscore}int16{\textunderscore}t fin} -- флаг указывает, что отправитель закончил
            передачу данных;
            \item \texttt{u{\textunderscore}int16{\textunderscore}t syn} -- флаг указывает, что установлена новое
            соединение;
            \item \texttt{u{\textunderscore}int16{\textunderscore}t rst} -- флаг указывает на необходимость прервать
            соединение;
            \item \texttt{u{\textunderscore}int16{\textunderscore}t psh} -- флаг указывает, что следует отправить данные
            без ожидания заполнения буфера;
            \item \texttt{u{\textunderscore}int16{\textunderscore}t ack} -- флаг указывает, что подтверждение номера
            последовательности действительно;
            \item \texttt{u{\textunderscore}int16{\textunderscore}t urg} -- флаг указывает, что присутствует важная
            информация, которая требует приоритетной обработки;
            \item \texttt{u{\textunderscore}int16{\textunderscore}t window} -- размер окна, который определяет количество
            данных, которые могут быть переданы отправителем до получения
            подтверждения;
            \item \texttt{u{\textunderscore}int16{\textunderscore}t check} -- контрольная сумма заголовка TCP протокола
            (проверка целостности данных);
            \item \texttt{u{\textunderscore}int16{\textunderscore}t urg{\textunderscore}ptr} -- указатель на полезную нагрузку в TCP-
            сегменте, которая требует приоритетной обработки.
        \end{itemize}}

    \item[6] {{\texttt{Struct udphdr}} -- системная структура данных, хранящая параметры UDP
    протокола. Содержит следующие поля:
        \begin{itemize}
            \item \texttt{u{\textunderscore}int16{\textunderscore}t source} -- порт отправителя;
            \item \texttt{u{\textunderscore}int16{\textunderscore}t dest} -- порт получателя;
            \item \texttt{u{\textunderscore}int16{\textunderscore}t len} -- общая длина заголовка UDP протокола и данных
            пакета в байтах;
            \item \texttt{u{\textunderscore}int16{\textunderscore}t check} -- контрольная сумма заголовка UDP протокола
            (проверка целостности данных).
        \end{itemize}}

    \item[7] {{\texttt{Struct icmphdr}} -- системная структура данных, хранящая параметры
    ICMP протокола. Содержит следующие поля:
    \begin{itemize}
        \item \texttt{u{\textunderscore}int8{\textunderscore}t type} -- тип ICMP-сообщения;
        \item \texttt{u{\textunderscore}int8{\textunderscore}t code} -- уточнения типа сообщения;
        \item \texttt{u{\textunderscore}int16{\textunderscore}t checksum} -- контрольная сумма заголовка ICMP
        протокола (проверка целостности данных);
        \item \texttt{u{\textunderscore}int16{\textunderscore}t id} -- идентификатор запроса (устанавливается
        отправителем, используется для передачи пакетов эхо-запроса и эхо-
        ответа);
        \item \texttt{u{\textunderscore}int16{\textunderscore}t} sequence -- порядковый номер запроса
        (устанавливается отправителем, используется для передачи пакетов эхо-
        запроса и эхо-ответа);
        \item \texttt{u{\textunderscore}int32{\textunderscore}t gateway} -- адрес шлюза (для сообщений, отправляемых
        через шлюз)\cite{unix_prof};
        \item \texttt{u{\textunderscore}int16{\textunderscore}t mtu} -- максимальный размер пакета, который может
        быть передан по маршруту.
    \end{itemize}}
\end{enumerate}